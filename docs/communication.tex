\documentclass[architecture.tex]{subfiles}
\begin{document}
The frontend communicates through \lstinline{fetch} calls, making RESTful
HTTP requests to the backend, typically situated on the same server.
The Flask server runs from \lstinline{backend/server.py}, where all endpoints
are defined.

Each call is made through POST, and accepts a dictionary of parameters as input.
The expected inputs are somewhat standardised e.g. when taking a move as input,
the endpoints all expect a key named \lstinline{move} containing a four-letter
move UCI: the square to move from, concatenated with the square to move to.

In the cases with more variation, at this point in time we refer to the
flask definitions as mentioned.

The backend holds an active board and related context-dependent computations
in \lstinline{backend/motor.py}. Here, the board is kept as well as a cursor
to the document in the database that is currently active.
Hence, the motor can fetch which moves are known (as theory or other moves)
easily. All interactions with the database are however relegated to
\lstinline{backend/database.py} where inserting, fetching and removing
are done, typically returning a document (in the form of cursor) or a
boolean status.

In addition to the context-holding motor, the \lstinline{backend/utils.py}
allows a developer to troubleshoot the database, add moves and boards
from sources such as PGN files or Polyglot files.
From here a \textit{crawling evaluation} can be started from a given
position, doing a recursive walk with a maximum depth, evaluating
every position with Stockfish.
\end{document}
